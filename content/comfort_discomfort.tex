\part{Theorie}\label{part:thorie}


\section{Das Modell vom Behagen und Unbehagen}\label{sec:das-modell-vom-behagen-und-unbehagen}

Wie unser Gehirn die Umwelt verarbeitet und wertet, kann im binären Modell von Joe Navarro von Behagen und Unbehagen,
oder auch Wohlsein und Unwohlsein modelliert werden.

Zu jeder Zeit in unserem Leben verspüren wir Gefühle und Empfindungen,
welche in die eine oder ander Kategorie des Modells eingegliedert werden können.
Unser Körper reagiert darauf auf neurochemischer Ebene, was unsere Stimmung und unser Verhalten bestimmt.
Diese Reaktionen auf Behagen und Unbehagen sind uns angeboren und waren schon für Urmenschen überlebenswichtig.
Wahrgenommene Informationen aus unserem Umfeld muss das Gehirn sofort verarbeiten und einordnen können,
um abschätzen zu können, ob eine Bedrohung vorliegt oder uns etwas Unbehagen verursachen kann.

Zur Veranschaulichung können wir einige Beispiele von Emotionen und Verhaltensweisen,
welche wir in Gegensatzpaare von Behagen und Unbehagen aufteilen können,
wie in der Tabelle~\textit{\nameref{tab:gegensatzpaare}} darstellen.

\begin{table}[htb]
    \centering
    \setupBfhTabular
    \begin{tabular}{ll}
        \rowcolor{BFH-tablehead}
        Zeichen für Behagen & Zeichen für Unbehagen \\\hline
        Gelassenheit            & Nervosität            \\\hline
        Zuversicht            & Zweifel \\\hline
        Freude & Verärgerung \\\hline
        Ruhe & Unruhe \\\hline
        Fröhlichkeit & Trauer \\\hline
        \multicolumn{2}{c}{\textellipsis}
    \end{tabular}
    \caption{Gegensatzpaare von Behagen und Unbehagen}
    \label{tab:gegensatzpaare}
\end{table}

Sind wir uns dieser Einordnung von Gefühlen und Verhaltensmuster in positive und negative Kategorien bewusst,
haben wir die Möglichkeit unser Umfeld bewusst wahrzunehmen und angemessen auf gewisse Situationen zu reagieren.