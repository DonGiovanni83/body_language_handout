\section{Einleitung}\label{sec:vorwort}
Noch bevor wir unsere erste Sprache erlernen, besitzen wir die Fähigkeit uns ganz ohne Worte zu verständigen.
Ob ein Säugling durch sein Weinen uns seine Unzufriedenheit mitteilt,
oder ein Fussballspieler mit erhobenen Armen sein Tor zelebriert,
wir Menschen besitzen die Fähigkeit solche universell verständlichen Kommunikationsweisen zu deuten.
Diese Arbeit analysiert diese Art von Kommunikation, oft als Körpersprache bezeichnet, auf ihre theoretischen Bausteine,
zeigt auf wie wir unser Umfeld bewusster deuten können und wie wir mit der eigenen Körpersprache unser Umfeld beeinflussen können.
\par
Der Inhalt dieser Arbeit basiert auf zwei Bücher geschrieben von Joe Navarro über die menschliche Körpersprache,
welche theoretischen Aufschluss wie auch praktische Ratschläge geben,
wie wir in unserem privaten wie auch im geschäftlichen Leben unser Umfeld deuten und beeinflussen können.\cite{menschen_verstehen_und_lenken,menschen_lesen}
