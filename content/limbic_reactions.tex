\section{Wie das limbische System auf das Umfeld reagiert}\label{sec:die-reaktionen-des-limbischen-systems}

Unser limbisches System hat im Laufe der menschlichen Entwicklung, drei überlebensnotwendige Reaktionen erlernt.
Diese Reaktionen haben einen wichtigen Einfluss auf unser Verhalten in Gefahrensituationen. 
Während zu Beginn der Menscheit urzeitliche Raubtiere eine Gefahr darstellten, sind es heutzutage alltägliche Stresssituationen, in denen 
der limbische Teil unseres Gehirns diese folgenden drei alternativen Reaktionen auslöst.

\begin{itemize}
    \item Schockstarre
    \item Flucht
    \item Kampf
\end{itemize}

\subsection{Schockstarre}
Wie bereits erwähnt haben wir das limbische System von unseren Vorfahren geerbt. Da die meisten Tiere bzw.
Raubtiere auf Bewegung reagieren (da Bewegung von Beutetieren bei diesen den Jagd-Instikt auslöst), war
die erste Verteidigungsstrategie unserer Vorfahren, wie auch von den meisten Tieren, angesichts einer solchen Gefahr in die Schockstarre 
zu fallen. Den meisten ist sicherlich auch bekannt, dass sich viele Tiere sogar tot stellen, wenn sie sich bedroht fühlen,
was ein Beispiel für eine extreme Form der Schockstarre ist. 
Heutzutage ist es jedoch sehr unwahrscheinlich, dass wir in unserem Alltag Raubtieren begegnen und trotzdem
gibt es moderne Alltagssituationen, in denen wir uns plötzlich bedroht oder verunsichert fühlen, auf die wir auf viel
subtilere Weise mit Schockstarre reagieren können. 
So kann man beispielsweise oft beobachten, wie Menschen oft erstarren, wenn sie bei einer
verbotenen oder geheimen Tat ertappt werden. Ein weiteres Beispiel ist, wie wir plötzlich kurz verharren, wenn wir
merken, dass wir etwas vergessen haben. Dieser kurzer Moment der Schockstarre verschafft uns Zeit schnell darüber nachzudenken,
wie wir vorgehen wollen (wie wir auf eine Bedrohungen oder Stresssituation am Besten reagieren müssen). 


\subsection{Flucht}
In gewissen gefährlichen Situationen, kann es sein, dass die Schockstarre ungeeignet ist oder nicht ausreicht als Reaktion,
um einer Gefahr zu entkommen. Das limbische System hat für solche Fälle eine weitere Reaktion auf Bedrohungen erlernt;
die Flucht. Viele Tiere, wie auch unserer Vorfahren, mussten ohne zu Überlegen als instiktive Reaktion in gewissen Situtationen
schnellst möglich fliehen und von Bedrohungen, wie Raubtieren, davonlaufen. Da wir jedoch heutzutage
nicht mehr in der Wildnis leben, sehen moderne Bedrohungen etwas anders aus und von denen laufen wir meistens nicht weg.
Doch auch wenn die Verhaltensweise bei einer "Flucht-Reaktion" sich nicht mehr so extrem zeigt, können wir sie heute noch beobachten.
Sich von Dingen oder Personen abwenzuwenden, sich einem unangehnehmen Gespräch zu entziehen, räumlichen Abstand zu gewinnen, den Blick abzuwenden, etwas vor sich
(beispielsweise beim Sitzen auf den Schoss) zu legen, sind gute Beispiele dafür.
Mit solchen Abwehrgesten versuchen wir uns von Dingen oder Personen in Stresssituationen zu distanzieren oder abzugrenzen und diese können wir
als nonverbale Signale in einer Kommunikation beobachten.


\subsection{Kampf}
Die letzte und radikalste Reaktion vom unserem limbischem System als Überlebenstaktik ist der Kampf.
D.h. wenn alles andere nicht hilft, gehen wir über in den Angriff, dabei verwandeln wir Angst in Wut um.
Da jedoch das Ausleben von Wut als Gewalt in unserem heutigen Alltag nicht nur unangebracht sondern auch gesetzeswidrig ist,
haben wir moderne Formen entiwckelt einen "Angriff" auszuüben: Streit, Schimpfwörter, Anschuldigungen, Provokationen und ähnliches.
So kann es sein, dass Menschen "aggressiv" reagieren, wenn sie sich bedroht fühlen oder verunsichert fühlen in gewissen Stresssituationen.
Den Kampf-Zustand zeigen viele auch (bewusst oder unbewusst) auch in ihrer Körpersprache: die Körperhaltung
verändert sich, beispielsweise streckt man die Brust aus oder setzt einen strengen Blick ein. 