\section{Definition: Nonverbale Kommunikation}

Nonverbale Kommunikation, oft auch als Körpersprache bezeichnet,
ist eine Form der Informationsvermittlung, welche nicht durch wörtliche Sprache umgesetzt wird.
Nonverbale Kommunikation wird also mithilfe von Mimik, Gestik, Körperbewegungen und Körperhaltung, der Stimme und vom Aussehen ausgedrückt.
Etwa 60 bis 65 Prozent der gesamten zwischenmenschlichen Kommunikation ist nonverbal.
Die nonverbale Kommunikation ist grundsätzlich ehrlicher und zuverlässiger als die verbale Kommunikation,
da die verbalen Äusserungen beweusst formuliert werden, während man sich meistens nicht
bewusst ist, dass man auch nonverbal kommuniziert.
