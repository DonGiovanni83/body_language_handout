\subsubsection{10 Gebote zur Entschlüsselung nonverbaler Signale}
\hskip 12pt\relax\textbf{1. Gebot: Du sollst ein aufmerksamer Beobachter deiner Umgebung sein.}
Um die nonverbale Kommunikation verstehen zu können, muss man genau so aufmerksam beobachten, wie man zuhören muss, um die verbale Kommunikation zu verstehen.
Das Beobachten der nonverbalen Kommunikation ist kein passiver Akt, sondern ein geplantes Verhalten, wobei wir all unsere
Sinne benutzen müssen, nicht nur das Sehvermögen.

\par\textbf{2. Gebot: Du sollst kontextbezogen beobachten.}
Kontextbezogen beobachten, bedeutet den gesamten Kontext, in dem sich eine Handlung abspielt, zu berücksichten, wenn man nonverbale
Kommunikation richtig deuten und verstehen möchte. Nonverbale ausgedrückte Angst von einer Person, die einen Verkehrsunfall hatte, muss man anders deuten,
als die nonverbal ausgedrückte Angst eines Verdächtigen in einem Verhör als Reaktion auf eine bestimmte Frage.

\par\textbf{3. Gebot: Du sollst lernen, universell gültige nonverbale Verhaltensweisen zu erkennen.}
Es gibt gewisse nonverbale Ausdrucksformen, die von den meisten Menschen ähnllich verwendet werden.
Diese gelten als universell und werden auch als "universelle Tells" bezeichnet. Ein Beispiel dafür ist das Zusammenpressen der Lippen, das Stress andeutet.

\par\textbf{4. Gebot: Du sollst lernen, idiosynkratische nonverbale Verhaltensweisen zu erkennen und zu deuten.}
Neben universellen nonverbalen Verhaltensweisen, gibt es noch die idiosynkratische nonverbale Verhaltensweisen.
Diese werden bei jedem Menschen mehr oder weniger individuell verwendet. Um diese erkennen zu können
sollte man Verhaltensmuster des Gegenübers beobachten. Wenn man mit den Verhaltensmuster des Gegenübers vertraut ist, 
erkennt man sie künftig wieder und kann sie besser deuten. 

\par\textbf{5. Gebot: Du sollst im Umgang mit anderen versuchen, ihr Normalverhalten zu ermitteln.}
Es ist wichtig, dass man eine Vorstellung vom Normalverhalten der Menschen, mit denen man regelmässig zu tun hat, bekommt. 
Dieses Normalverhalten muss also ermittelt werden, d.h. man muss beobachten was für eine Körperhaltung diese Menscen normalerweise haben.
Beispielsweise was sie für einen Gesichtsausdruck haben im Normalzustand, wo sie ihre Füsse hinstellen, wo sie ihre Hände ablegen oder wie sie sitzen und vieles mehr.
Das ist deshalb wichtig, weil man Abweichungen vom Normalverhalten nur dann erkennen kann und deuten kann, wenn man auch das Normalverhalten kennt.

\par\textbf{6. Gebot: Du sollst versuchen, immer nach multiplen Tells ausschau zu halten - Verhaltensweisen, die in Kombination oder in Folge auftreten.}
Meistens ist ein einziges Signal vom Gegenüber nicht ausreichend, um seine Körpersprache korrekt deuten zu können, 
weshalb man immer auf multiple Tells auschaue halten sollte. Das bedeutet erst wenn man mehrere körperliche Verhaltensweisen beobachtet, erkennt man 
ein einheitliches Gesamtbild, dass man dann viel zuverlässiger und auschlussreicher deuten kann.

\par\textbf{7. Gebot: Du sollst nach Verhaltensänderungen Ausschau halten, die auf eine Veränderung der Gedanken, Gefühle, Interessen oder Absichten hinweisen.}
Plötzliche Veränderungen des Verhaltens können darauf hinweisen, dass jemand Informationen verarbeitet oder auf emotionale Ereignisse reagiert.
Wenn man am Telefon eine schlechte Nachricht erhält oder etwas Belastendes sieht, spiegelt die Körpersprache diese Veränderung sofort wider. 
Unter gewissen Umständen können Veränderungen im Verhalten eines Menschen auch ein Interesse oder eine bestimmte Absicht offenbaren. 
Die genaue Beobachtung solcher Änderungen versetzt Sie somit in die Lage, Handlungen und Reaktionen einer Person gewissermaßen vorherzusehen. \cite{menschen_lesen}
 

\par\textbf{8. Gebot: Du sollst lernen, falsche oder irreführende nonverbale Signale zu erkennen.}
Um Menschen lesen zu können oder nonverbale Signale richtig deuten zu können, muss man sein Gegenüber
nicht nur bewusst und aufmerksam beobachten sondern man muss diese Signale auch sorgfälltig beurteilen, 
d.h. zwischen authentischen und irreführenden Hinweisen unterscheiden und somit einschätzen, ob das Gegenübers aufricht ist oder nicht.

\par\textbf{9. Gebot: Du sollst den Unterschied zwischen Behagen und Unbehagen erkennen.}
Eine wichtige Methode nonverbales Verhalten zu deuten, ist sich zu fragen, ob das Verhalten eines Menschen Behagen (Zufriedenheit, Entspannung, Freude, ...) 
oder Unbehagen (Unzufriedenheit, Stress, Anspannung, ...) ausdrückt. Dies hilft dabei das Gegenüber besser einzuschätzen. Meistens ist es 
möglich nonverbale Verhalten einer der beiden Kategorieren zuzuordnen.


\par\textbf{10. Gebot: Du sollst diskret sein, wenn du andere beobachtest.}
Das planvollen Beobachten der Menschen, um ihr nonverbales Verhalten deuten zu können, ist sehr wichtig, jedoch muss man unbedingt
vermeiden diese Absicht zu offenbaren. D.h. man muss aufmerksam und planvoll beobachten, aber sollte dabei immer unaufdringlich, unauffällig und diskret bleiben.
