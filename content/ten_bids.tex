\subsubsection{10 Gebote zur Entschlüsselung nonverbaler Signale}
\hskip 12pt\relax\textbf{1. Gebot: Du sollst ein aufmerksamer Beobachter deiner Umgebung sein.}
Um die nonverbale Kommunikation verstehen zu können, muss man genau so aufmerksam beobachten, wie man zuhören muss, um die verbale Kommunikation zu verstehen.
Das Beobachten der nonverbalen Kommunikation ist kein passiver Akt, sondern ein geplantes Verhalten, wobei wir all unsere
Sinne benutzen müssen, nicht nur das Sehvermögen.

\par\textbf{2. Gebot: Du sollst kontextbezogen beobachten.}
Kontextbezogen beobachten, bedeutet den gesamten Kontext, in dem sich eine Handlung abspielt, zu berücksichten, wenn man nonverbale
Kommunikation richtig deuten und verstehen möchte. Nonverbale ausgedrückte Angst von einer Person, die einen Verkehrsunfall hatte, muss man anders deuten,
als die nonverbal ausgedrückte Angst eines Verdächtigen in einem Verhör als Reaktion auf eine bestimmte Frage.

\par\textbf{3. Gebot: Du sollst lernen, universell gültige nonverbale Verhaltensweisen zu erkennen.}
Es gibt gewisse nonverbale Ausdrucksformen, die von den meisten Menschen ähnllich verwendet werden.
Diese gelten als universell und werden auch als "universelle Tells" bezeichnet. Ein Beispiel dafür ist das Zusammenpressen der Lippen, das Stress andeutet.

\par\textbf{4. Gebot: Du sollst lernen, idiosynkratische nonverbale Verhaltensweisen zu erkennen und zu deuten.}
Neben universellen nonverbalen Verhaltensweisen, gibt es noch die idiosynkratische nonverbale Verhaltensweisen.
Diese werden bei jedem Menschen mehr oder weniger individuell verwendet. Um diese erkennen zu können
sollte man Verhaltensmuster des Gegenübers beobachten. Wenn man mit den Verhaltensmuster des Gegenübers vertraut ist, 
erkennt man sie künftig wieder und kann sie besser deuten. 

\par\textbf{5. Gebot: Du sollst im Umgang mit anderen versuchen, ihr Normalverhalten zu ermitteln.}
\par\textbf{6. Gebot: Du sollst versuchen, immer nach multiplen Tells ausschau zu halten - Verhaltensweisen, die in Kombination oder in Folge auftreten.}
\par\textbf{7. Gebot: Du sollst nach Verhaltensänderungen Ausschau halten, die auf eine Veränderung der Gedanken, Gefühle, Interessen oder Absichten hinweisen.}
\par\textbf{8. Gebot: Du sollst lernen, falsche oder irreführende nonverbale Signale zu erkennen.}
\par\textbf{9. Gebot: Du sollst den Unterschied zwischen Behagen und Unbehagen erkennen.}
\par\textbf{10. Gebot: Du sollst diskret sein, wenn du andere beobachtest.}
